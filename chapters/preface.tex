\chapter*{คำนำ}
\addcontentsline{toc}{chapter}{คำนำ}

PHP เป็นหนึ่งในภาษาที่ได้รับความนิยมในการพัฒนา Web Application ถึงแม้ว่าจะมีความเห็นเกี่ยวกับความยืดหยุ่นเกินไปของภาษาอยู่บ้าง
แต่ภาษา PHP ก็มีการพัฒนาและปรับปรุงเรื่อยมาจนถึงปัจจุบัน ซึ่งด้วยความยืดหยุ่นของภาษานี้ก็ทำให้ภาษา PHP เป็นหนึ่งในภาษาที่เหมาะสำหรับผู้เริ่มต้น
แต่ก็ปฏิเสธไม่ได้ที่ความยืดหยุ่นนั้น จะทำให้รูปแบบการเขียนโปรแกรมด้วยภาษา PHP มีความหลากหลายเกินไป 
ดังนั้นจึงมีกลุ่มผู้พัฒนาที่มีแนวคิดในการออกแบบโครงสร้างของการพัฒนาที่เหมือนกันในรูปแบบของ Framework ซึ่ง Framework หนึ่งของภาษา PHP 
ที่ได้รับความนิยมอย่างมากในขณะนี้ คือ Laravel เนื่องจากมี community ขนาดใหญ่ และมี documentation ที่กระชับและเข้าใจง่าย

หนังสือเล่มนี้มีวัตถุประสงค์ให้ผู้อ่านมีความรู้และความเข้าใจในหลักการพัฒนา Web Application ด้วย Laravel Framework โดยหนังสือเล่มนี้
จะบอกเล่าพื้นฐานของ Framework เป็นขั้นตอน เพื่อให้เข้าใจภาพรวมของการพัฒนา และนำไปประยุกต์ใช้ได้ต่อไป

หนังสือเล่มนี้ได้รับการปรับปรุงจากเอกสารที่ใช้ประกอบการอบรม Laravel Framework เมื่อเดือนมิถุนายน 2562 ขณะนั้นเป็น Laravel version 5.8
และในขณะที่เริ่มเขียนหนังสือเล่มนี้ คือ เดือนกันยายน 2563 ซึ่ง version ของ Laravel คือ Laravel 8.0

\hspace*{\fill} ศรชัย ลักษณะปีติ

\hspace*{\fill} ภาควิชาวิทยาการคอมพิวเตอร์ 

\hspace*{\fill} คณะวิทยาศาสตร์ มหาวิทยาลัยเกษตรศาสตร์ 

\hspace*{\fill} กันยายน 2563