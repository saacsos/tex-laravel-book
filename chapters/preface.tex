\chapter*{คำนำ}
\addcontentsline{toc}{chapter}{คำนำ}

คุณภาพของซอฟต์แวร์มีความสำคัญมากในปัจจุบัน เนื่องจากซอฟต์แวร์ได้เข้ามาเป็นส่วนหนึ่งของชีวิตประจำวันของทุกคน ผู้พัฒนาซอฟต์แวร์จึงควรคำนึงถึงคุณภาพของซอฟต์แวร์เสมอ การทดสอบซอฟต์แวร์เป็นวิธีการหนึ่งที่จะช่วยยกระดับคุณภาพของซอฟต์แวร์ให้ดีขึ้น เป็นวิธีที่ทำได้ง่าย
แม้แต่ผู้ที่เพิ่งเริ่มพัฒนาซอฟต์แวร์ก็สามารถทดสอบโปรแกรมได้ ซึ่งส่วนใหญ่จะทำโดยการพิมพ์ผลการทำงานของโปรแกรมออกมาทางหน้าจอเพื่อตรวจสอบว่า ผลการทำงานถูกต้องตามที่คาดหวังไว้หรือไม่ อย่างไรก็ตาม การทดสอบในลักษณะนี้อาจทำโดยใช้สัญชาตญาณในการทดสอบและไม่มีหลักการที่ชัดเจน  

หนังสือเล่มนี้มีวัตถุประสงค์ให้ผู้อ่านมีความรู้และความเข้าใจในหลักการทดสอบซอฟต์แวร์ ทั้งทฤษฎีการทดสอบและการนำทฤษฎีเหล่านั้นมาประยุกต์ใช้ในการทดสอบซอฟต์แวร์ รวมถึงแนะนำการออกแบบโปรแกรมเชิงวัตถุที่จะช่วยให้สามารถทดสอบได้สะดวก รวดเร็ว และถูกต้องมากยิ่งขึ้น
นอกจากนั้น การทดสอบโดยการรันโปรแกรมและตรวจดูผลการทำงานเองนั้นช้าและมีโอกาสผิดพลาดได้ง่าย จึงควรนำเครื่องมือการทดสอบที่เหมาะสมมาใช้ หนังสือเล่มนี้จึงอธิบายการใช้เครื่องมือช่วยในการทดสอบซอฟต์แวร์ด้วย เพื่อให้การทดสอบเป็นไปอย่างอัตโนมัติมากยิ่งขึ้น อนึ่ง เครื่องมือการทดสอบในหนังสือเล่มนี้เป็นเครื่องมือสำหรับโปรแกรมภาษาจาวา (Java) ผู้อ่านจึงควรมีพื้นฐานการโปรแกรมภาษาจาวา เพื่อให้เข้าใจการใช้งานเครื่องมือได้ง่ายขึ้น


\hspace*{\fill} ศรชัย ลักษณะปีติ

\hspace*{\fill} ภาควิชาวิทยาการคอมพิวเตอร์ 

\hspace*{\fill} คณะวิทยาศาสตร์ มหาวิทยาลัยเกษตรศาสตร์ 

\hspace*{\fill} กันยายน 2563